\chapter{Dataset overview}
In this section, we'll explore the datasets utilized within this thesis. Our focus will be on 2D front-facing and profile human binary silhouettes. This form was chosen upon the data provided by the BodyM dataset. The subjects are positioned in an a-pose, ensuring greater consistency in the samples.
\section{SURREACT}
\subsubsection{Description}
SURREACT \cite{surreact} is a synthetic dataset built on SMPL model. The main goal of the work was to explore benefits of using synthetic data for  human action recognition.  The study aimed to answer whether the synthetic data could potentially improve accuracy of already existing methods. This theory was confirmed and even shown improvements over other state-of-the-art action recognition methods. This is however not as important for this thesis as we are not going to use the features that were added.

The dataset introduced by ~\cite{super} is an extension of the SURREACT dataset, incorporating the data generation techniques and a custom annotation method. This thesis utilizes a modified version of this dataset. The original dataset comprises 50,000 human scans, meshes, annotations, and other data of subjects in the T-Pose. In contrast, our customized version offers 79,999 frontal and 79,999 lateral images with annotations, featuring subjects in the A-Pose.  They are saved in RGBA format with dimensions of 320x240 without background thus eliminating the need of segmentation. Measurements are saved in .npy file format requiring us to use NumPy~\cite{numpy} to read these values.

\subsubsection{Measurements}
The measurements provided by this dataset are rather clearly described.

\subsubsection{SMPL}

\section{BodyM}
\subsubsection{Description}
This public body measurement dataset~\cite{BodyM} contains measurement and image data from real human subjects. The subjects were photographed in a well-lit indoor setup, resulting in the data being less prone to segmentation inaccuracies. Subjects also wore tight-fitting clothing to better reflect the measurements. After the pictures were taken, the subjects were scanned using Treedy photogrammetric scanner and fitted to the SMPL mesh. Measurements were then taken on said meshes.
\subsubsection{Measurements}
\subsubsection{Issues}