\chapter{Implementation}

\section{Used software}
\subsubsection{Keras}
Keras~\cite{keras} is an open-source neural network library written in Python. Thanks to its user-friendly interface and modular design is Keras one of the leading frameworks in neural network development. Its simple yet flexible architecture allows for easy prototyping and experimentation, making it an ideal choice for both beginners and experienced practitioners in the field of deep learning.
\subsubsection{OpenCV }
Open Source Computer Vision Library (OpenCV for short)~\cite{opencv} is a comprehensive open-source library originally developed by Intel. It is mainly used for various tasks in fields such as computer vision or machine learning. At the time of writing this thesis, OpenCV provides over 2500 optimized algorithms. These can effectively perform many tasks such as face detection, object tracking, image preprocessing and many more. Providing interfaces in multiple programming languages such as Python, C++, Java and MATLAB it is very popular with the community as well as recognisable and famous companies.


%The output tensor is then passed through ReLU~\cite{relu} along with batch optimization. Subsequently, a max pooling layer is used followed by a  convolutional layer. The output is then once again processed by max pooling layer with configuration same as before.  The result is then flattened to a tensor which is passed to a fully connected layer and a ReLU. As the last layer a regressor is used to provide the measurements estimation.


%\section{Setting up network}
%\section{Figuring out measurements}
%\section{Optimalizations}
%\subsection{Including BodyM in training}
%\subsection{Adding height}
%\subsection{Adding lateral image}
%\subsection{Combinations}

