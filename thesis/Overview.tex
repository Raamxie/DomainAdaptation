\chapter{Overview}
\section{Traditional measuring methods}
Human body can be measured using many different approaches. All of these have their advantages and disadvantages. In this thesis we only need to introduce two of them.
\subsection{Hand measurement}
This is the traditional method of using tape measure for obtaining measurements. This approach usually needs one extra person that performs the measurement on the subject. This way the measurements are more precise than the subject using the tape measure without help. The measurements however have to be taken at specific locations to provide correct information in further processing. The locations vary depending on the use and thus there is no universal guide for their measurement. Whereas the professionals are familiar with them, the subjects are usually not as informed. This creates space for human error we are trying to avoid.
\subsection{Digital measuring on 3D model}
This method is mentioned because BodyM dataset uses this method to provide ground truth values. Is it based on photogrammetry scanner which creates precise meshes of scanned object. After the human body get acquired from the scanner they are registered to SMPL mesh topology. We can then use the resulting mesh for calculating the measurements.
\section{Obstacles}
One of the issues when working with human body measurements is the lack of real world data. The process of measuring is time-consuming and requires privacy measures to take place to protect subjects' personal information. This can be avoided by using synthetic datasets.\\
Moreover the time complexity to train a neural network can be reduced by using powerful device which is not available.
\section{Problem specification}

\section{SMPL}
SMPL is short for Skinned Multi-Person Linear model. This model is based on skinning and blend shapes. The model has been trained on thousands of 3D body scans to create a realistic human model. Full explanation of functions and workflow of this model is out of scope of this work.

\section{Neural network}
Neural network is code built on premises of how human brains work. It consists of connected nodes called neurons. Each neuron takes input variables, processes them and then sends the result to other neurons. Every connection has associated weight which determines the influence said value will have.\\
Neurons are then organised into layers. They are usually divided into input, output and hidden layers. The function of the hidden layers is to perform the operations needed to calculate output from the input data.\\
The process of training adjusts the weights of the connections. This automatic process of adjusting is usually based on comparing the output and correct value we provide for the network and minimising the difference.\\
This process helps the network to find complex relationships or patterns that may not  be as understandable for humans. \\\\
Our model is mainly built on the following layers:
\subsubsection{Convolution layer}
The convolution operation consists of moving small filter called kernel along the image computing the dot product between kernel and input producing activation map for said filter.  The values in kernel are learnable parameters meaning they are adjusted over the training process.
\subsubsection{Max Pooling layer}
Max pooling is operation of down-sampling 
\subsubsection{ReLU}
\subsection{Linear regression}
In this thesis we will be using linear regression to predict human body measurements. The predictions are based on assumption of linear connection between variables. 

\section{Keras}
\section{OpenCV}