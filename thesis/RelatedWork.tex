\chapter{Related Work}
%\section{Automatic Estimation of Anthropometric Human Body Measurements}
%This thesis builds on foundations laid out by article from \cite{super}. This article delves into estimating body measurements on different data types. As this thesis works with 2D data only, we will use only relevant parts. Moreover, the article provides us with annotations for synthetic dataset by Surreact \cite{surreact}. 
The need of knowing human measurements is always present. The methods used for obtaining have been developing and nowadays are competing with traditional measurement methods in precision. The progress has been developing also thanks to new ways of obtaining data used for these estimations. First, only numerical values were available to researchers. Available measurements were used for statistical models which were used to roughly estimate average measurements.


%The solution proposed in \cite{HBDE1} provides us with a method which estimates subject's height using single uncalibrated image. Another approach  \cite{KeepItSMPL} uses joints position estimation to create a 3D mesh used for further evaluation.
\section{History}
Quetelet's concept of the average man has had a profound influence on the development of psychology and the statistical study of human characteristics~\cite{adolphe}. Quetelet argued that measurements of human traits would conform to the normal distribution, with the average representing the true or ideal type. This notion of the average as representative allowed early psychologists to blur the distinction between individual-level data and aggregate statistics.
The idea of the average man as a statistical model for understanding human nature persisted in psychology, even as reporting practices shifted away from individual-level results towards aggregate statistics. Quetelet's work laid the groundwork for the widespread use of large-scale data collection and analysis techniques, which became central to the field's pursuit of understanding individual differences and population-level phenomena.
The legacy of Quetelet's work highlights the important epistemic challenges that arise when connecting statistical models to claims about individuals and human nature. This historical context is relevant for understanding the development of methods for estimating human body measurements from data, which often rely on aggregate statistics and population-level modeling.

Since the 18th century, the military began employing anthropometric measurements, primarily focusing on stature, to identify suitable candidates~\cite{anthrohistory}.
...
\section{Pose Estimation}
Pose estimation is a computer vision task that involves detecting and locating key points on objects or persons in images or videos. These key points correspond to specific body parts or landmarks, such as joints. The goal is to accurately determine the spatial position and orientation (pose) of the person within the image or video frame. Pose estimation has various applications, including human-computer interaction, gesture recognition, action recognition, augmented reality, and robotics. It's a fundamental technology that enables machines to understand and interact with the physical world more intuitively.

\section{Domain Adaptation}
When training a machine learning model, such as a neural network, it is crucial for the performance of the model that the training and testing data are similar and follow the same distribution~\cite{domainAdaptation}. However, this condition is often not met. One reason for this is the limited amount of training data, as demonstrated in this thesis. Another reason is the slight differences between the test data and the data the network was trained on. For instance, this can occur with medical devices where output devices may have varying colour representations. Considering factors like time complexity and data availability, domain adaptation can be an optimal solution to address these discrepancies.

Domain adaptation is a type of transfer learning, used to mitigate domain shift between source (training data) and target (data used for testing) domain. Multiple approaches have been developed to address this issue:

\subsection{Instance-Based Methods}
 Instance-based methods reweigh or select instances from the source domain to better match the target domain's distribution. One of the notable approaches, Kernel Mean Matching~\cite{instanceMethod} (or KMM for short),  aims to adjust the distribution of the source domain by assigning weights to source instances. This process minimizes the difference between the  source and target distributions in a high-dimensional feature space.
 
 \subsection{Feature-Based Methods}
 Feature-based methods focus on learning a common feature representation that is invariant across domains. These methods typically involve transforming the feature space such that the source and target domains become indistinguishable. One important method is the Maximum Mean Discrepancy (MMD). DAN~\cite{featureMethod} incorporates MMD into a deep neural network, aligning the distributions of source and target features at multiple layers of the network. %Explain MMD
 
 Another influential method is Domain-Adversarial Neural Networks (DANN)~\cite{featureMethod2}. DANN employs adversarial training, where a domain classifier is trained to be able to distinguish source and target features while the feature extractor simultaneously learns to confuse the domain classifier. This adversarial process influences the feature extractor to generate domain-invariant features, enhancing model performance on the target domain.
 
 \subsection{Parameter-Based Methods}
 Parameter-based methods involve sharing or regularizing model parameters between the source and target domains. These methods often adapt the source model to the target domain by imposing regularization constraints. Deep Domain Confusion (DDC)~\cite{parameterMethod} combines a domain confusion loss with the standard classification loss. The domain confusion loss encourages the network to learn features that are not only discriminative for the source task but also invariant across domains. %needs better explanation what it really does

\subsection{Hybrid Methods}
Hybrid methods integrate multiple adaptation strategies to leverage their complementary strengths. CyCADA (Cycle-Consistent Adversarial Domain Adaptation)~\cite{hybridMethod} combines adversarial training, cycle-consistency, and feature alignment. CyCADA uses generative adversarial networks (GANs) to translate images from the source domain to the target domain and vice versa, ensuring that the translations are cycle-consistent. This approach helps align the feature distributions while preserving the semantic content, leading to improved adaptation performance.

\section{Neural Anthropometer}
An important article is the~\cite{source} which proposes a method to tackle this task. Its Neural Anthropometer provides a valuable approach which will we use as backbone our convolutional neural network architecture. We do not need everything used in this article as we already have annotated synthetic dataset provided by~\cite{super}. To keep the network as small as possible due to resource consumption and training difficulty increase with size. The proposed architecture starts with a binary image silhouette input. This is then processed by a convolutional layer. Number of channels was based on number of values on output. The output tensor is then passed through ReLU~\cite{relu} along with batch optimization. Subsequently, a max pooling layer is used followed by a  convolutional layer. The output is then once again processed by max pooling layer with configuration same as before.  The result is then flattened to a tensor which is passed to a fully connected layer and a ReLU. As the last layer a regressor is used to provide the measurements estimation.

This approach is further implemented by~\cite{super}. The article delves into different data representations of human bodies and influence this 
